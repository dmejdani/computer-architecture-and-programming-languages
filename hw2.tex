\documentclass[a4paper]{article}
\usepackage[pdftex]{hyperref}
\usepackage[latin1]{inputenc}
\usepackage[english]{babel}
\usepackage{a4wide}
\usepackage{amsmath}
\usepackage{amssymb}
\usepackage{algorithmic}
\usepackage{algorithm}
\usepackage{ifthen}
\usepackage{listings}
% move the asterisk at the right position
\lstset{basicstyle=\ttfamily,tabsize=4,literate={*}{${}^*{}$}1}
%\lstset{language=C,basicstyle=\ttfamily}
\usepackage{moreverb}
\usepackage{palatino}
\usepackage{multicol}
\usepackage{tabularx}
\usepackage{comment}
\usepackage{verbatim}
\usepackage{color}

%% personal packages
\usepackage[super]{nth}
\usepackage{enumitem}
\usepackage[nodayofweek]{datetime}
\usepackage{circuitikz}
\usepackage{karnaugh-map}

%% new date format
\newdateformat{mydate}{\twodigit{\THEDAY}{ }\shortmonthname[\THEMONTH], \THEYEAR}

%% pdflatex?
\newif\ifpdf
\ifx\pdfoutput\undefined
\pdffalse % we are not running PDFLaTeX
\else
\pdfoutput=1 % we are running PDFLaTeX
\pdftrue
\fi
\ifpdf
\usepackage{graphicx}
\else
\usepackage{graphicx}
\fi
\ifpdf
\DeclareGraphicsExtensions{.pdf, .jpg}
\else
\DeclareGraphicsExtensions{.eps, .jpg}
\fi

\parindent=0cm
\parskip=0cm

\setlength{\columnseprule}{0.4pt}
\addtolength{\columnsep}{2pt}

\addtolength{\textheight}{5.5cm}
\addtolength{\topmargin}{-26mm}
\pagestyle{empty}

%%
%% Sheet setup
%% 
\newcommand{\coursename}{Computer Architecture and Programming Languages}
\newcommand{\courseno}{CO20-320241}
 
\newcommand{\sheettitle}{Homework}
\newcommand{\mytitle}{}
\newcommand{\mytoday}{\today}


% Current Assignment number
\newcounter{assignmentno}
\setcounter{assignmentno}{2}

% Current Problem number, should always start at 1
\newcounter{problemno}
\setcounter{problemno}{1}

%%
%% problem and bonus environment
%%
\newcounter{probcalc}
\newcommand{\problem}[2]{
  \pagebreak[2]
  \setcounter{probcalc}{#2}
  ~\\
  {\large \textbf{Problem {\arabic{assignmentno}}.{\arabic{problemno}}} \hspace{0.2cm}\textit{#1}} \refstepcounter{problemno}\vspace{2pt}\\}

\newcommand{\bonus}[2]{
  \pagebreak[2]
  \setcounter{probcalc}{#2}
  ~\\
  {\large \textbf{Bonus Problem \textcolor{blue}{\arabic{assignmentno}}.\textcolor{blue}{\arabic{problemno}}} \hspace{0.2cm}\textit{#1}} \refstepcounter{problemno}\vspace{2pt}\\}

%% some counters  
\newcommand{\assignment}{\arabic{assignmentno}}

%% solution  
\newcommand{\solution}{\pagebreak[2]{\bf Solution:}\\}

%% Hyperref Setup
\hypersetup{pdftitle={Homework \assignment},
  pdfsubject={\coursename},
  pdfauthor={},
  pdfcreator={},
  pdfkeywords={Computer Architecture and Programming Languages},
  %  pdfpagemode={FullScreen},
  %colorlinks=true,
  %bookmarks=true,
  %hyperindex=true,
  bookmarksopen=false,
  bookmarksnumbered=true,
  breaklinks=true,
  %urlcolor=darkblue
  urlbordercolor={0 0 0.7}
}

\begin{document}
\coursename \hfill Course: \courseno\\
Jacobs University Bremen \hfill \mytoday\\
{Desar Mejdani}\hfill
\vspace*{0.3cm}\\
\begin{center}
{\Large \sheettitle{} {\assignment}\\}
\end{center}

\textbf{Note:} Every number which is missing its base in this homework it is meant as a number in base 10.

\problem{}{0}
\solution
\begin{enumerate}[label=(\alph*)]
	\item $777_8 + 1 = 1000_8$
	\item $888_{16} + 1 = 889_{16}$
	\item $32007_8 + 1 = 32010_8$
	\item $32108_{16} + 1 = 32109_{16}$
	\item $8BFF_{16} + 1 = 8C00_{16}$
	\item $1219_{16} + 1 = 121A_{16}$
\end{enumerate}


\problem{}{0}
\solution
\begin{enumerate}[label=(\alph*)]
	\item $\mathrm{x} = \overline{\mathrm{A}\cdot \mathrm{B}\cdot (\mathrm{C}+\mathrm{D})}$
	
	\begin{circuitikz}
		\draw
		(0,2) node[and port] (myand1) {}
		(0,0) node[or port] (myor1) {}
		(2,1) node[and port] (myand2) {}
		(3,1) node[not port] (mynot1) {}
		(myand1.in 1) node[left]{A}
		(myand1.in 2) node[left]{B}
		(myor1.in 1) node[left]{C}
		(myor1.in 2) node[left]{D}
		(myand1.out) -- (myand2.in 1)
		(myor1.out) -- (myand2.in 2)
		(myand2.out) -- (mynot1.in)
		
		(mynot1.out) node[right] {x}
		;
	\end{circuitikz}

	\item $\mathrm{x = \overline{A + B + \overline{C} \cdot D \cdot \overline{E}} + \overline{B} \cdot C\cdot \overline{D}}  $
	\\
	\\
	\begin{circuitikz}
		\draw
		(0,2) node[or port] (abor) {}
		(-.5,0) node[not port] (cnot) {}
		(2,-1) node[and port] (ncdand) {}
		(4,-2) node[and port] (neand) {}
		(-.5,-2.3) node[not port] (enot) {}
		(-.5, -4.3) node[not port] (dnot) {}
		(6,0) node[or port] (myor1) {}
		(1,3) node[not port] (bnot) {}
		(8.5,1.5) node[and port] (myand1) {}
		(6.5,3.28) node[and port] (myand2) {}
		(10.5,0.28) node[or port] (lastor) {}
		
		(abor.in 1) node[left] (A)  [anchor=east,xshift=-1cm] {A}
		(abor.in 2) node[left] (B)  [anchor=east,xshift=-1cm] {B}
		(cnot.in) node[left] (C)  [anchor=east,xshift=-1cm] {C}
		(ncdand.in 2) node[left](D)  [anchor=east,xshift=-2.8cm] {D}
		(enot.in) node[left](E)  [anchor=east,xshift=-1cm] {E}
		(lastor.out) node[right](Output) {x}
		
		(abor.in 1) -- (A)
		(abor.in 2) -- (B)
		(cnot.in) -- (C)
		(ncdand.in 2) -- (D)
		(enot.in) -- (E)
		(-2, 3) -| (bnot.in)
		(-2, 3) to[short, -*] (-2,1.72)
		(-1.7, 4) -| (myand2.in 1)
		(-1.7, 4) to[short, -*] (-1.7,0)
		(-1.7,-4.3) -| (dnot.in)
		(-1.7,-4.3) to[short, -*] (-1.7, -1.3)
		(bnot.out) -- (myand2.in 2)
		(myand2.out) -| (myand1.in 1)
		(dnot.out) -| (myand1.in 2)
		(myor1.out) -- (lastor.in 2)
		(myand1.out) -| (lastor.in 1)
		
		
		
		
		(cnot.out) -| (ncdand.in 1)
		(ncdand.out) |- (neand.in 1)
		(enot.out) -- (neand.in 2)
		(abor.out) -| (myor1.in 1)
		(neand.out) -| (myor1.in 2)
	
		;
		
	\end{circuitikz}
	\newpage
	
	\item 
	$\mathrm{x = \overline{W+P\cdot \overline{Q}}}$
	
	\begin{circuitikz}
		\draw
		
		(0,0) node[not port] (qnot) {}
		(5,2) node[or port] (myor) {}
		(3,1) node[and port] (myand) {}
		(7,2) node[not port] (lastnot) {}
		
		(myor.in 1) node[left] (W)  [anchor=east,xshift=-5cm] {W}
		(myand.in 1) node[left] (P) [anchor=east,xshift=-3cm] {P}
		(qnot.in) node[left] (Q) [anchor=east, xshift=-.6cm] {Q}
		(lastnot.out) node[right] () {x}		
		(W) -- (myor.in 1)
		(P) -- (myand.in 1)
		(Q) -- (qnot.in)
		
		(qnot.out) -| (myand.in 2)
		(myand.out) -| (myor.in 2)
		(myor.out) -- (lastnot.in)
		
		;
	\end{circuitikz}

	\item 
	$\mathrm{x=(A+B)\cdot (\overline{A}+\overline{B})}$
	
	\begin{circuitikz}
		\draw
		(0,0) node[or port] (myor1) {}
		(1,3) node[not port] (mynot1) {}
		(1,1) node[not port] (mynot2) {}
		(4,2) node[or port] (myor2) {}
		(6,.28) node[and port] (myand) {}
		
		(myor1.in 1) node[left] (A) [xshift=-1cm] {A}
		(myor1.in 2) node[left] (B) [xshift=-1cm] {B}
		(myand.out) node[right] {x}
		
		(A) -- (myor1.in 1)
		(B) -- (myor1.in 2)
		
		(-2,3) -- (mynot1.in)
		(-2,3) to[short, -*] (-2,.28)
		(-1.5,1) -- (mynot2.in)
		(-1.5,1) to[short, -*] (-1.5, -.28)
		
		(myor1.out) -- (myand.in 2)
		(mynot1.out) -| (myor2.in 1)
		(mynot2.out) -| (myor2.in 2)
		(myor2.out) -| (myand.in 1)
		
		
		;
	\end{circuitikz}
	
\end{enumerate}

\problem{}{0}
\solution
\begin{itemize}
	\item Truth table
	
	\begin{table}[!ht]
		\centering
		\begin{tabular}{|c|c|c|c|}
			\hline
			M & N & Q & X      \\ \hline
			0 & 0 & 0 & 0      \\ \hline
			0 & 0 & 1 & 0      \\ \hline
			0 & 1 & 0 & 0      \\ \hline
			0 & 1 & 1 & 1      \\ \hline
			1 & 0 & 0 & 0      \\ \hline
			1 & 0 & 1 & 1      \\ \hline
			1 & 1 & 0 & 0      \\ \hline
			1 & 1 & 1 & 1      \\ \hline
		\end{tabular}
	\end{table}
\item Sum of products expression: $\mathrm{x=\overline{M}NQ + M\overline{N}Q + MNQ}$
\item Simplification:
$\mathrm{x=\overline{M}NQ + M\overline{N}Q + MNQ}$\\
\begin{align*}
&\implies \mathrm{Q(\overline{M}N + M\overline{N}+ MN)}\,\, \text{{Distributive Rule}}\\
&\implies \mathrm{Q(\overline{M}N + M\overline{N}+ MN)}\,\, \text{{Distributive Rule}}\\
&\implies \mathrm{Q(\overline{M}N + M(\overline{N}+N))}\,\, \text{Distributive Rule}\\
&\implies \mathrm{Q(\overline{M}N+M)}\,\, \text{Complement Rule}\\
&\implies \mathrm{Q((M+\overline{M})(M+N))}\,\, \text{Distributive Rule}\\
&\implies \mathrm{Q(M+N)}\,\, \text{Complement Rule}\\
&\implies \mathrm{QM+QN}\,\, \text{Distributive Rule}\\
\end{align*}

\end{itemize}
\newpage

\problem{}{0}
\solution
\begin{enumerate}[label=(\alph*)]
	\item $\mathrm{X+\overline{X} \cdot Y = X + Y}$
	\begin{itemize}
		\item	$\mathrm{LeftSide = X+\overline{X} \cdot Y}$
		\begin{table}[!ht]
			\centering
			\begin{tabular}{|c|c|c|c|}
				\hline
				X & Y & $\mathrm{\overline{X}Y}$ & Output \\ \hline
				0 & 0 & 0                       & 0      \\ \hline
				0 & 1 & 1                       & 1      \\ \hline
				1 & 0 & 0                       & 1      \\ \hline
				1 & 1 & 0                       & 1      \\ \hline
			\end{tabular}
		\end{table}

		\item $\mathrm{RightSide = X + Y}$
		\begin{table}[!ht]
			\centering
			\begin{tabular}{|c|c|c|l|}
				\hline
				X & Y & \multicolumn{2}{c|}{Output} \\ \hline
				0 & 0 & \multicolumn{2}{c|}{0}      \\ \hline
				0 & 1 & \multicolumn{2}{c|}{1}      \\ \hline
				1 & 0 & \multicolumn{2}{c|}{1}      \\ \hline
				1 & 1 & \multicolumn{2}{c|}{1}      \\ \hline
			\end{tabular}
		\end{table}	
		
	\end{itemize}
As we can see, the truth tables are the same.

\item $\mathrm{\overline{X} + X\cdot Y = \overline{X} + Y}$
	\begin{itemize}
		\item $\mathrm{LeftSide = \overline{X}+X\cdot Y}$
		
		\begin{table}[!ht]
			\centering
			\begin{tabular}{|c|c|c|c|c|}
				\hline
				X & Y & $\mathrm{\overline{X}}$ & $\mathrm{X \cdot Y}$  & Output \\ \hline
				0 & 0 & 1                       & 0                     & 1      \\ \hline
				0 & 1 & 1                       & 0                     & 1      \\ \hline
				1 & 0 & 0                       & 0                     & 0      \\ \hline
				1 & 1 & 0                       & 1                     & 1      \\ \hline
			\end{tabular}
		\end{table}
	
	\item $\mathrm{RightSide=\overline{X}+Y}$
	\begin{table}[!ht]
		\centering
		\begin{tabular}{|c|c|c|c|}
			\hline
			X & Y & $\mathrm{\overline{X}}$ & Output \\ \hline
			0 & 0 & 1                       & 1      \\ \hline
			0 & 1 & 1                       & 1      \\ \hline
			1 & 0 & 0                       & 0      \\ \hline
			1 & 1 & 0                       & 1      \\ \hline
		\end{tabular}
	\end{table}
	\end{itemize}

As we can see, the truth tables are the same.
\end{enumerate}

\problem{}{0}
\solution
\begin{enumerate}[label=(\alph*)]
	\item $\mathrm{A+1 = 1}$
	\item $\mathrm{A\cdot A = A}$
	\item $\mathrm{B\cdot \overline{B} = 0}$
	\item $\mathrm{C+C = C}$
	\item $\mathrm{x\cdot 0 = 0}$
	\item $\mathrm{D\cdot 1 = D}$
	\item $\mathrm{D + 0 = D}$
	\item $\mathrm{C + \overline{C} = 1}$
	\item $\mathrm{G+G\cdot F = G(1+F)=G}$	
	\item $\mathrm{y+\overline{w}\cdot y = y(1+\overline{w}) = y}$
\end{enumerate}
\newpage

\problem{}{0}
\solution

\begin{table}[!ht]
	\centering
	\begin{tabular}{|c|c|c|c|c|c|}
		\hline
		w & y & $\mathrm{\overline{w+y}}$ & $\mathrm{\overline{w}\, \overline{y}}$ & $\mathrm{\overline{wy}}$ & \begin{tabular}[c]{@{}c@{}}$\mathrm{\overline{w}+ \overline{y} }$\end{tabular} \\ \hline
		0 & 0 & 1                         & 1                                   & 1                        & 1                                                                                  \\ \hline
		0 & 1 & 0                         & 0                                   & 1                        & 1                                                                                  \\ \hline
		1 & 0 & 0                         & 0                                   & 1                        & 1                                                                                  \\ \hline
		1 & 1 & 0                         & 0                                   & 0                        & 0                                                                                  \\ \hline
	\end{tabular}
\end{table}
As we can see, $\mathrm{\overline{w+y} = \overline{w}\,\overline{y}}$, and $\mathrm{\overline{wy}= \overline{w}+\overline{y}}$.


\problem{}{0}
\solution

Sum of products expression: $\mathrm{X = \overline{A}\,\overline{B}\,\overline{C}D + \overline{A}\,\overline{B}CD+ \overline{A}B\overline{C}D + \overline{A}BCD + A\overline{B}\,\overline{C}\,\overline{D}+ AB\overline{C}D+ABCD}$

\begin{align*}
	&\text{Rearranging to work on consecuitive two terms}\\
	\implies &\mathrm{
		\overline{A}\,\overline{B}\,\overline{C}D + \overline{A}\,\overline{B}CD+ \overline{A}BCD +ABCD + \overline{A}B\overline{C}D + AB\overline{C}D +  A\overline{B}\,\overline{C}\,\overline{D}}\\
	&\text{Distributive Rule - Here applied on two consecutive terms}\\
	\implies &\mathrm{\overline{A}\,\overline{B}(D(\overline{C}+C))+ BCD(\overline{A} + A) + BD(\overline{C}(\overline{A} + A)) + A\overline{B}\,\overline{C}\,\overline{D}}\\
	&\text{Complement Rule}\\
	\implies &\mathrm{\overline{A}\,\overline{B}D+ BCD+BD\overline{C} + A\overline{B}\,\overline{C}\,\overline{D}}\\
	&\text{Distributive Rule}\\
	\implies &\mathrm{\overline{A}\,\overline{B}D + BD(\overline{C} + C) + A\overline{B}\,\overline{C}\,\overline{D}}\\
	&\text{Complement Rule}\\
	\implies &\mathrm{\overline{A}\,\overline{B}D + BD + A\overline{B}\,\overline{C}\,\overline{D}}\\
	&\text{Distributive Rule}\\
	\implies &\mathrm{D(\overline{A}\,\overline{B} + B) + A\overline{B}\,\overline{C}\,\overline{D}}\\
	&\text{Distributive Rule}\\
	\implies &\mathrm{D((\overline{A}+B)(\overline{B} + B)) + A\overline{B}\,\overline{C}\,\overline{D}}\\
	&\text{Complement Rule}\\
	\implies &\mathrm{D(\overline{A}+B) + A\overline{B}\,\overline{C}\,\overline{D}}\\
	&\text{Distributive Rule}\\
	\implies &\mathrm{BD + \overline{A}D + A\overline{B}\,\overline{C}\,\overline{D}}\\
	\end{align*}

\problem{}{0}
\solution
Expression:  $\mathrm{X = \overline{A}\,\overline{B}\,\overline{C}D + \overline{A}\,\overline{B}CD+ \overline{A}B\overline{C}D + \overline{A}BCD + A\overline{B}\,\overline{C}\,\overline{D}+ AB\overline{C}D+ABCD}$

\begin{centering}


\begin{karnaugh-map}[4][4][1][$AB$][$CD$]
	\minterms{4, 12, 5, 13, 2, 7, 15}
	\maxterms{0,1,3,6,8,9,10,11,14}
	\implicant{2}{2}
	\implicant{4}{13}
	\implicant{5}{15}
\end{karnaugh-map}

\end{centering}
We can see from the Karnaugh-map that we have two groups of 4, which we can use to simplify the expression. For instance, in the green group $\mathrm{B}$ and $\mathrm{\overline{B}}$ are both used, and as well, $\mathrm{C}$ and $\mathrm{\overline{C}}$. Thus, these inputs can be removed from the expression, leaving: $\mathrm{\overline{A}D}$.

Similarly, in the yellow group $\mathrm{A}$ and $\mathrm{\overline{A}}$, and , $\mathrm{C}$ and $\mathrm{\overline{C}}$ are all used. Thus, these inputs can be removed from the expression too, leaving: $\mathrm{BD}$.

For the red group, we cannot use it to simplify the expression further, so we are left with: $\mathrm{A\overline{B}\,\overline{C}\,\overline{D}}$.	

The overall simplified expression is: $$\mathrm{X = \overline{A}D + BD + A\overline{B}\,\overline{C}\,\overline{D}}$$
\end{document}
