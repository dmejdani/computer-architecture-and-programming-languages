\documentclass[a4paper]{article}
\usepackage[pdftex]{hyperref}
\usepackage[latin1]{inputenc}
\usepackage[english]{babel}
\usepackage{a4wide}
\usepackage{amsmath}
\usepackage{amssymb}
\usepackage{algorithmic}
\usepackage{algorithm}
\usepackage{ifthen}
\usepackage{listings}
% move the asterisk at the right position
\lstset{basicstyle=\ttfamily,tabsize=4,literate={*}{${}^*{}$}1}
%\lstset{language=C,basicstyle=\ttfamily}
\usepackage{moreverb}
\usepackage{palatino}
\usepackage{multicol}
\usepackage{tabularx}
\usepackage{comment}
\usepackage{verbatim}
\usepackage{color}

%% personal packages
\usepackage[super]{nth}
\usepackage{enumitem}
\usepackage[nodayofweek]{datetime}
\usepackage{circuitikz}
\usepackage{karnaugh-map}

%% new date format
\newdateformat{mydate}{\twodigit{\THEDAY}{ }\shortmonthname[\THEMONTH], \THEYEAR}

%% pdflatex?
\newif\ifpdf
\ifx\pdfoutput\undefined
\pdffalse % we are not running PDFLaTeX
\else
\pdfoutput=1 % we are running PDFLaTeX
\pdftrue
\fi
\ifpdf
\usepackage{graphicx}
\else
\usepackage{graphicx}
\fi
\ifpdf
\DeclareGraphicsExtensions{.pdf, .jpg}
\else
\DeclareGraphicsExtensions{.eps, .jpg}
\fi

\parindent=0cm
\parskip=0cm

\setlength{\columnseprule}{0.4pt}
\addtolength{\columnsep}{2pt}

\addtolength{\textheight}{5.5cm}
\addtolength{\topmargin}{-26mm}
\pagestyle{empty}

%%
%% Sheet setup
%% 
\newcommand{\coursename}{Computer Architecture and Programming Languages}
\newcommand{\courseno}{CO20-320241}

\newcommand{\sheettitle}{Homework}
\newcommand{\mytitle}{}
\newcommand{\mytoday}{\today}


% Current Assignment number
\newcounter{assignmentno}
\setcounter{assignmentno}{7}

% Current Problem number, should always start at 1
\newcounter{problemno}
\setcounter{problemno}{1}

%%
%% problem and bonus environment
%%
\newcounter{probcalc}
\newcommand{\problem}[2]{
	\pagebreak[2]
	\setcounter{probcalc}{#2}
	~\\
	{\large \textbf{Problem {\arabic{assignmentno}}.{\arabic{problemno}}} \hspace{0.2cm}\textit{#1}} \refstepcounter{problemno}\vspace{2pt}\\}

\newcommand{\bonus}[2]{
	\pagebreak[2]
	\setcounter{probcalc}{#2}
	~\\
	{\large \textbf{Bonus Problem \textcolor{blue}{\arabic{assignmentno}}.\textcolor{blue}{\arabic{problemno}}} \hspace{0.2cm}\textit{#1}} \refstepcounter{problemno}\vspace{2pt}\\}

%% some counters  
\newcommand{\assignment}{\arabic{assignmentno}}

%% solution  
\newcommand{\solution}{\pagebreak[2]{\bf Solution:}\\}

%% Hyperref Setup
\hypersetup{pdftitle={Homework \assignment},
	pdfsubject={\coursename},
	pdfauthor={},
	pdfcreator={},
	pdfkeywords={Computer Architecture and Programming Languages},
	%  pdfpagemode={FullScreen},
	%colorlinks=true,
	%bookmarks=true,
	%hyperindex=true,
	bookmarksopen=false,
	bookmarksnumbered=true,
	breaklinks=true,
	%urlcolor=darkblue
	urlbordercolor={0 0 0.7}
}

\begin{document}
\coursename \hfill Course: \courseno\\
Jacobs University Bremen \hfill \mytoday\\
{Desar Mejdani}\hfill
\vspace*{0.3cm}\\
\begin{center}
	{\Large \sheettitle{} {\assignment}\\}
\end{center}

\problem{}{0}
\solution
\begin{verbatim}
my_function:
    slti $t0, $a0, 11      # test if x < 11
    bne $t0, $zero, ELSE   # goto ELSE if x < 11
    sub $v0, $a0, $a1      # x - y
    jr $ra                 # jump back to calling routine

ELSE:
    add $v0, $a0, $a1      # x + y
    jr $ra                 # jump back to calling routine

\end{verbatim}


\problem{}{0}
\solution
\begin{verbatim}
is_more_than_fifty:
    addi $sp, $sp, -4      # decrementing stack pointer
    sw $ra, 0($sp )        # storing return address in the stack
    jal prod               # jump and link to prod
    lw $ra, 0($sp)         # restoring the return address
    addi $sp, $sp, 4       # incrementing stack pointer
    slti $t0, $v0, 51      # comparing if smaller than 51
    bne $t0, $zero, ELSE   # if smaller than 51 goto ELSE
    addi $v0, $zero, 1     # set return value to 1
    jr $ra                 # return to caller

prod:
    mul $v0, $a0, $a1      # a * b
    jr $ra                 # return to caller

ELSE:
    add $v0, $zero, $zero  # set return value to 0
    jr $ra                 # return to caller
\end{verbatim}


\problem{}{0}
\solution


\begin{verbatim}
Loop: sll $t1, $s3, 2     # index multiplied by 4, (4bytes)
      add $t1, $t1, $s6   # adding the base address of A
      lw $t0, 0($t1)      # loading A[i]
      beq $t0, $s5, Exit  # comparing with -1 and branching if equal
      addi $s3, $s3, 1    # incrementing index
      j Loop
Exit:

//======== C Code ==========

// int i = 0;  // assumed that i is zero at the beginning of the loop
while(A[i] != -1)
{
i++;
}
\end{verbatim}

\newpage
\begin{verbatim}

\end{verbatim}
\problem{}{0}
\solution

All the machine code for the following MIPS instructions are taken from the following table fig.\ref{fig:circuit}, found in the lecture slides.

 \begin{figure}[!ht]
	\centering
	\includegraphics[scale=.5]{table.png}
	\caption{Reference table for MIPS introductions}
	\label{fig:circuit}
\end{figure}

The register numbers are taken from the table below, also found in the slides, fig.\ref{fig:circuit2}.

 \begin{figure}[!ht]
	\centering
	\includegraphics[scale=.5]{table2.png}
	\caption{Reference table for register numbers}
	\label{fig:circuit2}
\end{figure}

Each instruction address differs from the one before or after by 4 bytes in its address (shown in the PC column).

For the branching instruction, the destination, if branching happens, should be at address 60024. We know that the destination is calculated with the following formula: 
$$\text{destination = PC + 4 + word address}$$
$$\implies \text{word address = destination - PC - 4}$$

Thus, the result is: $60024 - 60012 - 4 = 8$. We know that this is a word address, thus we have to divide by 4 bytes / word, we get two words. This, value is shown in the machine language code in the table below.

For the jump statement the address is a byte address. Thus, to jump back to 'LOOP' the jump address has to be 60000 / 4 = 15000.


\begin{table}[!htbp]
	\scalebox{0.84}{
	\begin{tabular}{|c|l|c|c|c|c|c|c|c|c|c|c|c|c|}
		\hline
		PC    & \multicolumn{1}{c|}{MIPS} & \multicolumn{6}{c|}{Machine Language (decimal)}   & \multicolumn{6}{c|}{Machine Language (binary)}                 \\ \hline
		60000 & Loop: sll \$t1, \$s3, 2     & 0      & 0      & 19     & 9      & 2     & 0     & 000000 & 00000 & 10011 & 01001      & 00010      & 000000      \\ \hline
		60004 & add \$t1, \$t1, \$s6        & 0      & 9      & 22     & 9      & 0     & 32    & 000000 & 01001 & 10110 & 01001      & 00000      & 100000      \\ \hline
		60008 & lw \$t0, 0(\$t1)            & 35     & 9      & 8      & \multicolumn{3}{c|}{0} & 100011 & 01001 & 01000 & \multicolumn{3}{c|}{0000000000000000} \\ \hline
		60012 & beq \$t0, \$s5, Exit        & 4      & 8      & 21     & \multicolumn{3}{c|}{2} & 000100 & 01000 & 10101 & \multicolumn{3}{c|}{0000000000000010} \\ \hline
		60016 & addi \$s3, \$s3, 1          & 8      & 19     & 19     & \multicolumn{3}{c|}{1} & 001000 & 10011 & 10011 & \multicolumn{3}{c|}{0000000000000001} \\ \hline
		60020 & j Loop                    & 2      & \multicolumn{5}{c|}{15000}               & 000010 & \multicolumn{5}{c|}{00000000000011101010011000}       \\ \hline
		60024 & Exit:                     & \multicolumn{6}{l|}{}                             & \multicolumn{6}{l|}{}                                          \\ \hline
	\end{tabular}
	}
\end{table}{\tiny }

\newpage

\problem{}{0}
\solution
Hexadecimal to binary and decimal.
\begin{verbatim}
0x0C000000 = 0000 1100 0000 0000 0000 0000 0000 0000 = 201326592

0xC4630000 = 1100 0100 0110 0011 0000 0000 0000 0000 = 3294822400
\end{verbatim}

\begin{enumerate}[label=(\alph*)]
\item Two's complement:
	\begin{enumerate}
	\item Since the first digit of the binary representation of the first bit pattern is 0, the number is positive. So the number is: $201326592_10$.
	\item Since the first digit of the binary representation of the first bit pattern is 1, the number is negative. The bits need to be flipped and added with 1: $0011 1011 1001 1100 1111 1111 1111 1111 + 1 = - 0011 1011 1001 1101 0000 0000 0000 0000 = - 1000144896_{10}$. 
	\end{enumerate}


\item Unsigned integer: We will need to just convert the number from binary to decimal.
	\begin{enumerate}
	\item $2^{26} + 2^{27} = 201326592_{10}$
	\item $2^31 + 2^30 + 2^26 + 2^21 + 2^22 + 2^17 + 2^16 = 3294822400_{10}$
	\end{enumerate}

\item MIPS instruction:
	\begin{enumerate}
	\item The first 6 bits from the op code of the instruction, which is 3. This is the op code of 'jal' instruction. Everything else, the address, is 0. Thus the instruction is:\\
	\begin{verbatim}
	jal 0
	\end{verbatim}
	
	\item The first 6 bits form the op code of the instruction, which is 49. This is the op code of 'lwc1' instruction. The registers are registers 3 (\$v3), and the offset is 0. Thus the instruction is:
	\begin{verbatim}
	lwc1 $v1, 0($v1)
	\end{verbatim}
	\end{enumerate}
\end{enumerate}

	
\end{document}

