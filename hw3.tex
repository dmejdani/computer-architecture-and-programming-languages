\documentclass[a4paper]{article}
\usepackage[pdftex]{hyperref}
\usepackage[latin1]{inputenc}
\usepackage[english]{babel}
\usepackage{a4wide}
\usepackage{amsmath}
\usepackage{amssymb}
\usepackage{algorithmic}
\usepackage{algorithm}
\usepackage{ifthen}
\usepackage{listings}
% move the asterisk at the right position
\lstset{basicstyle=\ttfamily,tabsize=4,literate={*}{${}^*{}$}1}
%\lstset{language=C,basicstyle=\ttfamily}
\usepackage{moreverb}
\usepackage{palatino}
\usepackage{multicol}
\usepackage{tabularx}
\usepackage{comment}
\usepackage{verbatim}
\usepackage{color}

%% personal packages
\usepackage[super]{nth}
\usepackage{enumitem}
\usepackage[nodayofweek]{datetime}
\usepackage{circuitikz}
\usepackage{karnaugh-map}

%% new date format
\newdateformat{mydate}{\twodigit{\THEDAY}{ }\shortmonthname[\THEMONTH], \THEYEAR}

%% pdflatex?
\newif\ifpdf
\ifx\pdfoutput\undefined
\pdffalse % we are not running PDFLaTeX
\else
\pdfoutput=1 % we are running PDFLaTeX
\pdftrue
\fi
\ifpdf
\usepackage{graphicx}
\else
\usepackage{graphicx}
\fi
\ifpdf
\DeclareGraphicsExtensions{.pdf, .jpg}
\else
\DeclareGraphicsExtensions{.eps, .jpg}
\fi

\parindent=0cm
\parskip=0cm

\setlength{\columnseprule}{0.4pt}
\addtolength{\columnsep}{2pt}

\addtolength{\textheight}{5.5cm}
\addtolength{\topmargin}{-26mm}
\pagestyle{empty}

%%
%% Sheet setup
%% 
\newcommand{\coursename}{Computer Architecture and Programming Languages}
\newcommand{\courseno}{CO20-320241}
 
\newcommand{\sheettitle}{Homework}
\newcommand{\mytitle}{}
\newcommand{\mytoday}{\today}


% Current Assignment number
\newcounter{assignmentno}
\setcounter{assignmentno}{3}

% Current Problem number, should always start at 1
\newcounter{problemno}
\setcounter{problemno}{1}

%%
%% problem and bonus environment
%%
\newcounter{probcalc}
\newcommand{\problem}[2]{
  \pagebreak[2]
  \setcounter{probcalc}{#2}
  ~\\
  {\large \textbf{Problem {\arabic{assignmentno}}.{\arabic{problemno}}} \hspace{0.2cm}\textit{#1}} \refstepcounter{problemno}\vspace{2pt}\\}

\newcommand{\bonus}[2]{
  \pagebreak[2]
  \setcounter{probcalc}{#2}
  ~\\
  {\large \textbf{Bonus Problem \textcolor{blue}{\arabic{assignmentno}}.\textcolor{blue}{\arabic{problemno}}} \hspace{0.2cm}\textit{#1}} \refstepcounter{problemno}\vspace{2pt}\\}

%% some counters  
\newcommand{\assignment}{\arabic{assignmentno}}

%% solution  
\newcommand{\solution}{\pagebreak[2]{\bf Solution:}\\}

%% Hyperref Setup
\hypersetup{pdftitle={Homework \assignment},
  pdfsubject={\coursename},
  pdfauthor={},
  pdfcreator={},
  pdfkeywords={Computer Architecture and Programming Languages},
  %  pdfpagemode={FullScreen},
  %colorlinks=true,
  %bookmarks=true,
  %hyperindex=true,
  bookmarksopen=false,
  bookmarksnumbered=true,
  breaklinks=true,
  %urlcolor=darkblue
  urlbordercolor={0 0 0.7}
}

\begin{document}
\coursename \hfill Course: \courseno\\
Jacobs University Bremen \hfill \mytoday\\
{Desar Mejdani}\hfill
\vspace*{0.3cm}\\
\begin{center}
{\Large \sheettitle{} {\assignment}\\}
\end{center}

\textbf{Note:} Every number which is missing its base in this homework it is meant as a number in base 10.

\problem{}{0}
\solution
Rules: R1 = Distributive, R2 = Complement, R3 = Associative, R4 = DeMorgan, R5 = Involution

\begin{enumerate}[label=(\alph*)]
	\item $\mathrm{x = (M+N)(\overline{M} + P)(\overline{N}+\overline{P})}$
	\begin{align*}
		&= \mathrm{(M\overline{M} + MP + N\overline{M} + NP)(\overline{N}+\overline{P})\quad R1}\\
		&= \mathrm{0 + (MP + N\overline{M} + NP)(\overline{N} + \overline{P})\quad R2}\\
		&= \mathrm{MP\overline{N} + MP\overline{P} + N\overline{M}\,\overline{N} + N\overline{M}\,\overline{P} + NP\overline{N} + NP\overline{P}\quad R1}\\
		&= \mathrm{MP\overline{N} + N\overline{M}\,\overline{P}\quad R2}
	\end{align*}
	
	\item $\mathrm{z = \overline{A}B\overline{C} + AB\overline{C} + B\overline{C}D}$
	\begin{align*}
		&= \mathrm{\overline{C}(\overline{A}B + AB + BD)\quad R1}\\
		&= \mathrm{\overline{C}(B(\overline{A} + A) + BD)\quad R1}\\
		&= \mathrm{\overline{C}(B + BD)\quad R2}\\
		&= \mathrm{\overline{C}(B(1 + D))\quad R1}\\
		&= \mathrm{\overline{C}B\quad R1}
	\end{align*}
	
	\item $\mathrm{x = \overline{(M + N + P)Q}}$
	\begin{align*}
		&= \mathrm{\overline{(M + N + P)} + \overline{Q}\quad R4}\\
		&= \mathrm{\overline{M+N}\,\overline{P} + \overline{Q}\quad R4}\\
		&= \mathrm{\overline{M}\, \overline{N}\,\overline{P} + \overline{Q}\quad R4}
	\end{align*}
	
	\item $\mathrm{z = \overline{ABC + DEF}}$
	\begin{align*}
		&= \mathrm{\overline{ABC}\,\overline{DEF}\quad R4}\\
		&= \mathrm{(\overline{AB} + \overline{C})(\overline{DE} + \overline{F})\quad R4}\\
		&= \mathrm{(\overline{A} + \overline{B} + \overline{C})(\overline{D} + \overline{E} + \overline{F})\quad R4}\\
		&= \mathrm{\overline{A}\,\overline{D} + \overline{A}\,\overline{E} + \overline{A}\,\overline{F} + \overline{B}\,\overline{D} + \overline{B}\,\overline{E} + \overline{B}\,\overline{F} + \overline{C}\,\overline{D} + \overline{C}\,\overline{E} + \overline{C}\,\overline{F}\quad R1}
	\end{align*}
	
	\item $\mathrm{z = \overline{A\overline{B} + C\overline{D} + EF}}$
	\begin{align*}
		&= \mathrm{\overline{A\overline{B} + C\overline{D}}\,\overline{EF}\quad R4}\\
		&= \mathrm{\overline{A\overline{B}}\,\overline{C\overline{D}}\,\overline{EF}\quad R4}\\
		&= \mathrm{(\overline{A} + B)(\overline{C} + D)(\overline{E} + \overline{F})\quad R4, R5}\\
		&= \mathrm{(\overline{A}\,\overline{C} + \overline{A}D + B\overline{C} + BD)(\overline{E} + \overline{F})\quad R1}\\
		&= \mathrm{\overline{A}\,\overline{C}\,\overline{E} + \overline{A}\,\overline{C}\,\overline{F} + \overline{A}D\overline{E} + \overline{A}D\overline{F} + B\overline{C}\,\overline{E} + B\overline{C}\,\overline{F} + BD\overline{E} + BD\overline{F}\quad R1}
	\end{align*}
	
	\item $\mathrm{z = \overline{\overline{A + B\overline{C}} + D\overline{(E + \overline{F})}}}$
	\begin{align*}
		&= \mathrm{\overline{\overline{A}\,\overline{B\overline{C}} + D\overline{E}\overline{F}}\quad R4}\\
		&= \mathrm{\overline{\overline{A}(\overline{B} + C) + D\overline{E}F}\quad R4}\\
		&= \mathrm{\overline{\overline{A+B\overline{C}}}\,\overline{D\overline{(E+F)}}\quad R4}\\
		&= \mathrm{(A+B\overline{C})(\overline{D} + \overline{\overline{(E+F)}})\quad R4}\\
		&= \mathrm{(A + B\overline{C})(\overline{D} + E + F)\quad R5}\\
		&= \mathrm{A\overline{D} + AE + AF + B\overline{C}\,\overline{D} + B\overline{C}E + B\overline{C}\,\overline{F}\quad R1}
	\end{align*}
\end{enumerate}


\problem{}{0}
\solution

Truth table of the logic circuit:

\begin{table}[!ht]
	\centering
	\begin{tabular}{ccccc}
		\multicolumn{1}{c|}{\textbf{A}} & \multicolumn{1}{c|}{\textbf{B}} & \multicolumn{1}{c|}{\textbf{C}} & \multicolumn{1}{c|}{\textbf{D}} & \textbf{x} \\ \hline
		0                               & 0                               & 0                               & 0                               & 1          \\ \hline
		0                               & 0                               & 0                               & 1                               & 1          \\ \hline
		0                               & 0                               & 1                               & 0                               & 0          \\ \hline
		0                               & 0                               & 1                               & 1                               & 1          \\ \hline
		0                               & 1                               & 0                               & 0                               & 0          \\ \hline
		0                               & 1                               & 0                               & 1                               & 0          \\ \hline
		0                               & 1                               & 1                               & 0                               & 0          \\ \hline
		0                               & 1                               & 1                               & 1                               & 0          \\ \hline
		1                               & 0                               & 0                               & 0                               & 1          \\ \hline
		1                               & 0                               & 0                               & 1                               & 1          \\ \hline
		1                               & 0                               & 1                               & 0                               & 0          \\ \hline
		1                               & 0                               & 1                               & 1                               & 0          \\ \hline
		1                               & 1                               & 0                               & 0                               & 0          \\ \hline
		1                               & 1                               & 0                               & 1                               & 0          \\ \hline
		1                               & 1                               & 1                               & 0                               & 0          \\ \hline
		1                               & 1                               & 1                               & 1                               & 0          \\ \hline
	\end{tabular}
\end{table}

We can write the product sum as:
$$\mathrm{x = \overline{A}\,\overline{B}\,\overline{C}\,\overline{D} + \overline{A}\,\overline{B}\,\overline{C}D + \overline{A}\,\overline{B}CD + A\overline{B}\,\overline{C}\,\overline{D} + A\overline{B}\,\overline{C}D}$$

Thus, one can write the KarnaughMap as:

\begin{centering}
	
\begin{karnaugh-map}[4][4][1][$AB$][$CD$]
	\terms{0, 4, 12, 2, 6}{1}
	\terms{1, 3, 5, 7, 8, 9, 10, 11, 13, 14, 15}{0}
	\implicantedge{0}{4}{2}{6}
	\implicant{4}{12}
\end{karnaugh-map}

\end{centering}

One can see that in the red section the following are used: $\mathrm{A, \overline{A}, D, \overline{D}}$. Thus, the remaining term is: $\mathrm{\overline{B}\,\overline{C}}$.

In the green section the following are used: $\mathrm{C,\overline{C}}$. The remaining term is: $\mathrm{\overline{A}\,\overline{B}D}$.

In total: $$\mathrm{x = \overline{B}\,\overline{C} + \overline{A}\,\overline{B}D}$$
\newpage


\problem{}{0}
\solution

Waveform of Q. Note that the level of Q does not change (it stays 1 as it was).
\begin{figure}[!htbp]
	\centering
	\includegraphics[scale=.35]{3_3.png}
	\caption{Output Q}
	\label{fig:circuit2}
\end{figure}

State table of the flip-flop:

\begin{table}[!ht]
	\centering
	\begin{tabular}{|c|c|c|c|}
		\hline
		\textbf{S} & \textbf{R} & \textbf{CLK} & \textbf{Q} \\ \hline
		0          & 0          & UP           & no change  \\ \hline
		1          & 0          & UP           & 1          \\ \hline
		0          & 1          & UP           & 0          \\ \hline
		1          & 1          & UP           & Ambiguous  \\ \hline
	\end{tabular}
\end{table}

\problem{}{0}
\solution
Waveform of Q.
\begin{figure}[!htbp]
	\centering
	\includegraphics[scale=.35]{3_4.png}
	\caption{Output Q}
	\label{fig:circuit3}
\end{figure}


State table of the flip-flop:

\begin{table}[!ht]
	\centering
	\begin{tabular}{|c|c|c|c|}
		\hline
		\textbf{S} & \textbf{R} & \textbf{CLK} & \textbf{Q} \\ \hline
		0          & 0          & DOWN           & no change  \\ \hline
		1          & 0          & DOWN           & 1          \\ \hline
		0          & 1          & DOWN           & 0          \\ \hline
		1          & 1          & DOWN           & Ambiguous  \\ \hline
	\end{tabular}
\end{table}
\newpage

\problem{}{0}
\solution
Waveform of Q.
\begin{figure}[!htbp]
	\centering
	\includegraphics[scale=.35]{3_5.png}
	\caption{Output Q}
	\label{fig:circuit3}
\end{figure}


State table of the flip-flop:

\begin{table}[!ht]
	\centering
	\begin{tabular}{|c|c|c|c|}
		\hline
		\textbf{J} & \textbf{K} & \textbf{CLK} & \textbf{Q} \\ \hline
		0          & 0          & UP           & no change  \\ \hline
		1          & 0          & UP           & 1          \\ \hline
		0          & 1          & UP           & 0          \\ \hline
		1          & 1          & UP           & Togles     \\ \hline
	\end{tabular}
\end{table}


\problem{}{0}
\solution

Waveform of Q.
\begin{figure}[!htbp]
	\centering
	\includegraphics[scale=.35]{3_6.png}
	\caption{Output Q}
	\label{fig:circuit4}
\end{figure}


State table of the flip-flop:

\begin{table}[!ht]
	\centering
	\begin{tabular}{|c|c|c|c|}
		\hline
		\textbf{J} & \textbf{K} & \textbf{CLK} & \textbf{Q} \\ \hline
		0          & 0          & DOWN           & no change  \\ \hline
		1          & 0          & DOWN           & 1          \\ \hline
		0          & 1          & DOWN           & 0          \\ \hline
		1          & 1          & DOWN           & Togles     \\ \hline
	\end{tabular}
\end{table}
\newpage

\problem{}{0}
\solution
\begin{enumerate}[label=(\alph*)]
	\item The first flip-flop outputs 1 when B goes high and A has already a value of high. Thus, A should be set high before B. The second flip-flop is set to high when C goes high, and J has already a value of high. Thus, J should be set High before C. 
	
	The sequence is A, then B, then C.
	
	\item 
	The start pulse can be used to reset the whole state of the flip-flops, and initialize the circuit. 
	
	\item Equivalent circuit using D flip-flops. 
	\begin{figure}[!htbp]
		\centering
		\includegraphics[scale=.35]{3_7.png}
		\caption{Using D flip-flops}
		\label{fig:circuit5}
	\end{figure}
	
\end{enumerate}

\end{document}
