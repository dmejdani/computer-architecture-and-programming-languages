\documentclass[a4paper]{article}
\usepackage[pdftex]{hyperref}
\usepackage[latin1]{inputenc}
\usepackage[english]{babel}
\usepackage{a4wide}
\usepackage{amsmath}
\usepackage{amssymb}
\usepackage{algorithmic}
\usepackage{algorithm}
\usepackage{ifthen}
\usepackage{listings}
% move the asterisk at the right position
\lstset{basicstyle=\ttfamily,tabsize=4,literate={*}{${}^*{}$}1}
%\lstset{language=C,basicstyle=\ttfamily}
\usepackage{moreverb}
\usepackage{palatino}
\usepackage{multicol}
\usepackage{tabularx}
\usepackage{comment}
\usepackage{verbatim}
\usepackage{color}

%% personal packages
\usepackage[super]{nth}
\usepackage{enumitem}
\usepackage[nodayofweek]{datetime}

%% new date format
\newdateformat{mydate}{\twodigit{\THEDAY}{ }\shortmonthname[\THEMONTH], \THEYEAR}

%% pdflatex?
\newif\ifpdf
\ifx\pdfoutput\undefined
\pdffalse % we are not running PDFLaTeX
\else
\pdfoutput=1 % we are running PDFLaTeX
\pdftrue
\fi
\ifpdf
\usepackage[pdftex]{graphicx}
\else
\usepackage{graphicx}
\fi
\ifpdf
\DeclareGraphicsExtensions{.pdf, .jpg}
\else
\DeclareGraphicsExtensions{.eps, .jpg}
\fi

\parindent=0cm
\parskip=0cm

\setlength{\columnseprule}{0.4pt}
\addtolength{\columnsep}{2pt}

\addtolength{\textheight}{5.5cm}
\addtolength{\topmargin}{-26mm}
\pagestyle{empty}

%%
%% Sheet setup
%% 
\newcommand{\coursename}{Computer Architecture and Programming Languages}
\newcommand{\courseno}{CO20-320241}
 
\newcommand{\sheettitle}{Homework}
\newcommand{\mytitle}{}
\newcommand{\mytoday}{\today}


% Current Assignment number
\newcounter{assignmentno}
\setcounter{assignmentno}{1}

% Current Problem number, should always start at 1
\newcounter{problemno}
\setcounter{problemno}{1}

%%
%% problem and bonus environment
%%
\newcounter{probcalc}
\newcommand{\problem}[2]{
  \pagebreak[2]
  \setcounter{probcalc}{#2}
  ~\\
  {\large \textbf{Problem {\arabic{assignmentno}}.{\arabic{problemno}}} \hspace{0.2cm}\textit{#1}} \refstepcounter{problemno}\vspace{2pt}\\}

\newcommand{\bonus}[2]{
  \pagebreak[2]
  \setcounter{probcalc}{#2}
  ~\\
  {\large \textbf{Bonus Problem \textcolor{blue}{\arabic{assignmentno}}.\textcolor{blue}{\arabic{problemno}}} \hspace{0.2cm}\textit{#1}} \refstepcounter{problemno}\vspace{2pt}\\}

%% some counters  
\newcommand{\assignment}{\arabic{assignmentno}}

%% solution  
\newcommand{\solution}{\pagebreak[2]{\bf Solution:}\\}

%% Hyperref Setup
\hypersetup{pdftitle={Homework \assignment},
  pdfsubject={\coursename},
  pdfauthor={},
  pdfcreator={},
  pdfkeywords={Computer Architecture and Programming Languages},
  %  pdfpagemode={FullScreen},
  %colorlinks=true,
  %bookmarks=true,
  %hyperindex=true,
  bookmarksopen=false,
  bookmarksnumbered=true,
  breaklinks=true,
  %urlcolor=darkblue
  urlbordercolor={0 0 0.7}
}

\begin{document}
\coursename \hfill Course: \courseno\\
Jacobs University Bremen \hfill \mytoday\\
{Desar Mejdani}\hfill
\vspace*{0.3cm}\\
\begin{center}
{\Large \sheettitle{} {\assignment}\\}
\end{center}

\textbf{Note:} Every number which is missing its base in this homework it is meant as a number in base 10.

\problem{}{0}
\solution
\begin{enumerate}[label=(\alph*)]
	\item $10100_2 = 2^4 + 2^2 = 16 + 4 = 20$
	\item $11011011 = 2^7 + 2^6 + 2^4 + 2^3 + 2^1 + 2^0 = 128 + 64 + 16 + 8 + 2 + 1 = 219$
	\item $001001001_2 = 2^6 + 2^3 + 2^0 = 64 + 8 + 1 = 73$
	\item $111111111111_2 = 2^{11} + 2^{10} + 2^9 + 2^8 + 2^7 + 2^6 + 2^5 + 2^4 + 2^3 + 2^2 + 2^1 + 2^0 = 4095$
	\item $75077_8 = 7*8^4 + 5*8^3 + 7*8^1+ 7 = 31295$
	\item $12101_3 = 3^4 + 2*3^3 + 3^2 + 1 = 145$
	\item $26601_7 = 2*7^4 + 6*7^3 + 6*7^2 + 1 = 7155$
	\item $431021_5 = 4*5^5 + 3*5^4 + 5^3 + 2*5 + 1 = 14511$
\end{enumerate}


\problem{}{0}
\solution
\begin{enumerate}[label=(\alph*)]
	\item $4272_{10} = 2^{12} + 2^7 + 2^5 + 2^4 = 1000010110000_2$
	\item $CBA_{16} = 12 * 16^2 + 11 * 16^1 + 10 * 16^0 = 12 * 2^8 + 11 * 2^4 + 10*2^0 = 110010111010_2$
	\item $B8C_{16} = 11 * 16^2 + 8 * 16^1 + 12 * 16^0 = 2956_{10}$
	\item $29D8_{16} = 2 * 16^3 + 9 * 16^2 + 13*16^1 + 8 = 10.712$
	\item 
	$8CE_{16} + 1 = 8CF_{16}$ \\
	$8CF_{16} + 1 = 8D0_{16}$ \\
	$8D0_{16} + 1 = 8D1_{16}$ \\
	$8D1_{16} + 1 = 8D2_{16}$ \\
	$8D2_{16} + 1 = 8D3_{16}$
	
\end{enumerate}

\problem{}{0}
\solution
\begin{enumerate}[label=(\alph*)]
	\item $732_{10} = 0111\ 0011\ 0010_{BCD}$
	\item Invalid BCD codes are numbers from 10-15 in binary:
	\begin{itemize}
		\item $1010$
		\item $1011$
		\item $1100$
		\item $1101$
		\item $1110$
		\item $1111$
	\end{itemize}
	\item $1001\ 0101\ 0110_{BCD} = 956_{10}$
	\item Letter: M
	\begin{itemize}
		\item $77 = 2^6 + 2^3 + 2^2 + 2^0 = 1001101_2$
		\item $77 = 4D_{16}$
	\end{itemize}
	\item Letter: m
	\begin{itemize}
		\item $109 = 2^6 + 2^5 + 2^3 + 2^2 + 2^0 = 1101101_2$
		\item $109 = 6D_{16}$
	\end{itemize}
\end{enumerate}

\problem{}{0}
\solution
\begin{enumerate}[label=(\alph*)]
	\item The AND gate provides low output in response to one or more low inputs.
	\item The OR gate provides low output only when all inputs are low.
\end{enumerate}

\problem{}{0}
\solution
AND gate
\begin{table}[!ht]
	\begin{tabular}{cccc}
		\multicolumn{1}{c|}{\textbf{Input1}} & \multicolumn{1}{c|}{\textbf{Input2}} & \multicolumn{1}{c|}{\textbf{Input3}} & \textbf{Output} \\ \hline
		0                                    & 0                                    & 0                                    & 0               \\ \hline
		0                                    & 0                                    & 1                                    & 0               \\ \hline
		0                                    & 1                                    & 0                                    & 0               \\ \hline
		0                                    & 1                                    & 1                                    & 0               \\ \hline
		1                                    & 0                                    & 0                                    & 0               \\ \hline
		1                                    & 0                                    & 1                                    & 0               \\ \hline
		1                                    & 1                                    & 0                                    & 0               \\ \hline
		1                                    & 1                                    & 1                                    & 1               \\ \hline
	\end{tabular}
\end{table}

\problem{}{0}
\solution
OR gate
\begin{table}[!ht]
	\begin{tabular}{ccccc}
		\multicolumn{1}{c|}{\textbf{Input1}} & \multicolumn{1}{c|}{\textbf{Input2}} & \multicolumn{1}{c|}{\textbf{Input3}} & \multicolumn{1}{c|}{\textbf{Input4}} & \textbf{Output} \\ \hline
		0                                    & 0                                    & 0                                    & 0                                    & 0               \\ \hline
		0                                    & 0                                    & 0                                    & 1                                    & 1               \\ \hline
		0                                    & 0                                    & 1                                    & 0                                    & 1               \\ \hline
		0                                    & 0                                    & 1                                    & 1                                    & 1               \\ \hline
		0                                    & 1                                    & 0                                    & 0                                    & 1               \\ \hline
		0                                    & 1                                    & 0                                    & 1                                    & 1               \\ \hline
		0                                    & 1                                    & 1                                    & 0                                    & 1               \\ \hline
		0                                    & 1                                    & 1                                    & 1                                    & 1               \\ \hline
		1                                    & 0                                    & 0                                    & 0                                    & 1               \\ \hline
		1                                    & 0                                    & 0                                    & 1                                    & 1               \\ \hline
		1                                    & 0                                    & 1                                    & 0                                    & 1               \\ \hline
		1                                    & 0                                    & 1                                    & 1                                    & 1               \\ \hline
		1                                    & 1                                    & 0                                    & 0                                    & 1               \\ \hline
		1                                    & 1                                    & 0                                    & 1                                    & 1               \\ \hline
		1                                    & 1                                    & 1                                    & 0                                    & 1               \\ \hline
		1                                    & 1                                    & 1                                    & 1                                    & 1               \\ \hline
	\end{tabular}
\end{table}



\end{document}
