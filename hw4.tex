
\documentclass[a4paper]{article}
\usepackage[pdftex]{hyperref}
\usepackage[latin1]{inputenc}
\usepackage[english]{babel}
\usepackage{a4wide}
\usepackage{amsmath}
\usepackage{amssymb}
\usepackage{algorithmic}
\usepackage{algorithm}
\usepackage{ifthen}
\usepackage{listings}
% move the asterisk at the right position
\lstset{basicstyle=\ttfamily,tabsize=4,literate={*}{${}^*{}$}1}
%\lstset{language=C,basicstyle=\ttfamily}
\usepackage{moreverb}
\usepackage{palatino}
\usepackage{multicol}
\usepackage{tabularx}
\usepackage{comment}
\usepackage{verbatim}
\usepackage{color}

%% personal packages
\usepackage[super]{nth}
\usepackage{enumitem}
\usepackage[nodayofweek]{datetime}
\usepackage{circuitikz}
\usepackage{karnaugh-map}

%% new date format
\newdateformat{mydate}{\twodigit{\THEDAY}{ }\shortmonthname[\THEMONTH], \THEYEAR}

%% pdflatex?
\newif\ifpdf
\ifx\pdfoutput\undefined
\pdffalse % we are not running PDFLaTeX
\else
\pdfoutput=1 % we are running PDFLaTeX
\pdftrue
\fi
\ifpdf
\usepackage{graphicx}
\else
\usepackage{graphicx}
\fi
\ifpdf
\DeclareGraphicsExtensions{.pdf, .jpg}
\else
\DeclareGraphicsExtensions{.eps, .jpg}
\fi

\parindent=0cm
\parskip=0cm

\setlength{\columnseprule}{0.4pt}
\addtolength{\columnsep}{2pt}

\addtolength{\textheight}{5.5cm}
\addtolength{\topmargin}{-26mm}
\pagestyle{empty}

%%
%% Sheet setup
%% 
\newcommand{\coursename}{Computer Architecture and Programming Languages}
\newcommand{\courseno}{CO20-320241}
 
\newcommand{\sheettitle}{Homework}
\newcommand{\mytitle}{}
\newcommand{\mytoday}{\today}


% Current Assignment number
\newcounter{assignmentno}
\setcounter{assignmentno}{4}

% Current Problem number, should always start at 1
\newcounter{problemno}
\setcounter{problemno}{1}

%%
%% problem and bonus environment
%%
\newcounter{probcalc}
\newcommand{\problem}[2]{
  \pagebreak[2]
  \setcounter{probcalc}{#2}
  ~\\
  {\large \textbf{Problem {\arabic{assignmentno}}.{\arabic{problemno}}} \hspace{0.2cm}\textit{#1}} \refstepcounter{problemno}\vspace{2pt}\\}

\newcommand{\bonus}[2]{
  \pagebreak[2]
  \setcounter{probcalc}{#2}
  ~\\
  {\large \textbf{Bonus Problem \textcolor{blue}{\arabic{assignmentno}}.\textcolor{blue}{\arabic{problemno}}} \hspace{0.2cm}\textit{#1}} \refstepcounter{problemno}\vspace{2pt}\\}

%% some counters  
\newcommand{\assignment}{\arabic{assignmentno}}

%% solution  
\newcommand{\solution}{\pagebreak[2]{\bf Solution:}\\}

%% Hyperref Setup
\hypersetup{pdftitle={Homework \assignment},
  pdfsubject={\coursename},
  pdfauthor={},
  pdfcreator={},
  pdfkeywords={Computer Architecture and Programming Languages},
  %  pdfpagemode={FullScreen},
  %colorlinks=true,
  %bookmarks=true,
  %hyperindex=true,
  bookmarksopen=false,
  bookmarksnumbered=true,
  breaklinks=true,
  %urlcolor=darkblue
  urlbordercolor={0 0 0.7}
}

\begin{document}
\coursename \hfill Course: \courseno\\
Jacobs University Bremen \hfill \mytoday\\
{Desar Mejdani}\hfill
\vspace*{0.3cm}\\
\begin{center}
{\Large \sheettitle{} {\assignment}\\}
\end{center}

\problem{}{0}
\solution

	We see that the last gate is an AND gate, so all of its inputs must high. We can immediately say that C has to be high. 
	
	Consequently, since C is high and the result of the XNOR gate must be high, we can say that B must be high too. 
	
	Since B is high and the result of XOR gate has to be high, then A must be low.
	
	The input condition to result into a high X is: $\mathrm{ABC = 011}$.
	
\problem{}{0}
\solution
\begin{enumerate}[label=(\alph*)]
	\item Truth table
	
	\begin{table}[!ht]
		\centering
		\begin{tabular}{|c|c|c|c|}
			\hline
			\textbf{A} & \textbf{B} & \textbf{C} & \textbf{Y} \\ \hline
			0          & 0          & 0          & 1          \\ \hline
			0          & 0          & 1          & 1          \\ \hline
			0          & 1          & 0          & 0          \\ \hline
			0          & 1          & 1          & 1          \\ \hline
			1          & 0          & 0          & 1          \\ \hline
			1          & 0          & 1          & 0          \\ \hline
			1          & 1          & 0          & 0          \\ \hline
			1          & 1          & 1          & 0          \\ \hline
		\end{tabular}
	\end{table}

	\item Sum of products:
	$$\mathrm{Y = \overline{A}\,\overline{B}\,\overline{C} + \overline{A}\,\overline{B}C + \overline{A}BC + A\overline{B}\,\overline{C}}$$
	
\end{enumerate}


\problem{}{0}
\solution
\textbf{Note:} Every binary number in this exercise is a 2's complement representation. Thus, the ones which have the first bit 0 are positive and the ones which have it 1 are negative.

\begin{enumerate}[label=(\alph*)]
	\item $+27 = 2^4 + 2^3 + 2^1 + 2^0 = 00011011_2$
	\item $+66 = 2^6 + 2^1 = 01000010_2$
	\item $-18 = -(2^4 + 2^1) = -(00010010_2)\implies \text{Flipping the bits} \implies 11101101_2 \implies \text{Adding 1} = 11101110_2$
	\item $+127 = 2^6 + 2^5 + 2^4 + 2^3 + 2^2 + 2^1 + 2^0 = 01111111_2$
	\item $-127 = -(01111111_2) \implies \text{Flipping the bits} \implies 10000000_2 \implies \text{Adding one} = 10000001_2$
	\item $-128 = -(2^7) = -(10000000_2) \implies \text{Flipping the bits and adding one} = 10000000_2$
	\item $+131$\\
	To represent this number in binary would require the eighth bit to be 1. Since we are using 2's complements, a positive number has the eighth bit 0. Thus, we need at least 9 bits to write this number in this form. 
	With eight bits this number cannot be written in 2's complement.
	\item $-7 = -(2^2 + 2^1 + 2^0) = -(00000111_2) \implies \text{fliiping the bits} = 11111000_2 \implies \text{Adding one} = 11111001_2 $
\end{enumerate}
\newpage
	
	
\problem{}{0}
\solution
\textbf{Note:} Every binary number in this exercise is a 2's complement representation. Thus, the ones which have the first bit 0 are positive and the ones which have it 1 are negative.\\
\textbf{Note:} Every number which does not have the base written to it, it is meant as a number in base 10.

\begin{enumerate}[label=(\alph*)]
	\item $00011000_2 = 2^4 + 2^3 = 24_{10}$
	\item $11110101_2 \implies \text{Flipping the bits} = 00001010_2 \implies \text{Adding one} = -(00001011_2) = -(2^3 + 2 + 1) = -11$
	\item $01011011_2 = 2^0 + 2^1 + 2^3 + 2^4 + 2^6 = 91$
	\item $10110110_2 \implies \text{Flipping the bits} = 01001001_2 \implies \text{Adding one} = -(01001010_") = - (2^1 + 2^3 + 2^6) = -74$
	\item $11111111_2 \implies \text{Flipping the bits} = 00000000_2 \implies \text{Adding one} = -(00000001_2) = -1$
	\item $01101111_2 = 2^0 + 2^1 + 2^2 + 2^3 + 2^5 + 2^6 = 111$
	\item $10000001_2 \implies \text{Flipping the bits} = 01111110_2 \implies \text{Adding one} = -(01111111_2) = -(2^0 + 2^1 + 2^2 + 2^3 + 2^4 + 2^5 + 2^6) = -127$
	\item $10000000_2 \implies \text{Flipping the bits} = 01111111_2 \implies \text{Adding one} = -(10000000_2) = -128$
\end{enumerate}


\problem{}{0}
\solution
\textbf{Note:} The numbers that are written in groups of four digits are BCD numbers. Every other number that does not have its base it is meant as a number in base 10.
\begin{enumerate}[label=(\alph*)]
	\item 
	$27 \xrightarrow{\text{BCD Code}} 0010\,0111$\\
	$36 \xrightarrow{\text{BCD Code}} 0011\,0110$\\
	
	\begin{tabular}{r}
		0010 0111 \\
		+ 0011 0110\\
		\hline
		0101 1101\\		
	\end{tabular}

	We can see that the right part of the result is bigger than 9 in decimal, thus we add 6 to it and subtract 16 to the result. 
	$$1101_2 + 0110_2 - 10000_2 = 0011_2$$
	As a result we have a carry of 1 to be added to the left BCD number.
	$$0101_2 + 1  = 0110_2$$
	
	The result is: $0110\,0011$, which is 63 in decimal. 
	
	\item 
	$73 \xrightarrow{\text{BCD Code}} 0111\,0011$\\
	$29 \xrightarrow{\text{BCD Code}} 0010\,1001$\\
	
	\begin{tabular}{r}
		0111 0011\\
		+ 0010 1001\\
		\hline
		1001 1100\\
	\end{tabular}

	We can see that the right part of the result is bigger than 9, thus we add 6 to it and subtract 16 to the result.
	$$1100_2 = 12 \implies 12 + 6 - 16 = 2 = 0010_2 $$
	As a result we have a carry of 1 to be added to the left BCD number.
	$$1001_2 + 1 = 1010_2$$
	We can see that this number is also bigger than 9. We perform the same operation as before:
	$$1010_2 = 10 \implies 10 +6 - 16 = 0 = 0000_2$$
	As a result we have a carry of 1 to be added to a third BCD number.\\
	The overall result is: $0001\,0000\,0010$ which is 102 in decimal.
	
\end{enumerate}
\newpage

\problem{}{0}
\solution

\begin{enumerate}[label=(\alph*)]
	\item From 0 to 255. There are no negatives in this representation.
	\item From -128 to 127, if we are using 2's complements. There are 256 numbers in total, containing the 0 too.
	\item From 0 to 2047. There are no negatives in this representation.
	\item From -1024 to 1023, if we are using 2's complements. There are 2048 numbers in total, containing the 0 too.
	\item From -32768 to 32767, if we are using 2's complements. There are 65536 numbers in total, including 0.
\end{enumerate}

\end{document}
