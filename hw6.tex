\documentclass[a4paper]{article}
\usepackage[pdftex]{hyperref}
\usepackage[latin1]{inputenc}
\usepackage[english]{babel}
\usepackage{a4wide}
\usepackage{amsmath}
\usepackage{amssymb}
\usepackage{algorithmic}
\usepackage{algorithm}
\usepackage{ifthen}
\usepackage{listings}
% move the asterisk at the right position
\lstset{basicstyle=\ttfamily,tabsize=4,literate={*}{${}^*{}$}1}
%\lstset{language=C,basicstyle=\ttfamily}
\usepackage{moreverb}
\usepackage{palatino}
\usepackage{multicol}
\usepackage{tabularx}
\usepackage{comment}
\usepackage{verbatim}
\usepackage{color}

%% personal packages
\usepackage[super]{nth}
\usepackage{enumitem}
\usepackage[nodayofweek]{datetime}
\usepackage{circuitikz}
\usepackage{karnaugh-map}

%% new date format
\newdateformat{mydate}{\twodigit{\THEDAY}{ }\shortmonthname[\THEMONTH], \THEYEAR}

%% pdflatex?
\newif\ifpdf
\ifx\pdfoutput\undefined
\pdffalse % we are not running PDFLaTeX
\else
\pdfoutput=1 % we are running PDFLaTeX
\pdftrue
\fi
\ifpdf
\usepackage{graphicx}
\else
\usepackage{graphicx}
\fi
\ifpdf
\DeclareGraphicsExtensions{.pdf, .jpg}
\else
\DeclareGraphicsExtensions{.eps, .jpg}
\fi

\parindent=0cm
\parskip=0cm

\setlength{\columnseprule}{0.4pt}
\addtolength{\columnsep}{2pt}

\addtolength{\textheight}{5.5cm}
\addtolength{\topmargin}{-26mm}
\pagestyle{empty}

%%
%% Sheet setup
%% 
\newcommand{\coursename}{Computer Architecture and Programming Languages}
\newcommand{\courseno}{CO20-320241}

\newcommand{\sheettitle}{Homework}
\newcommand{\mytitle}{}
\newcommand{\mytoday}{\today}


% Current Assignment number
\newcounter{assignmentno}
\setcounter{assignmentno}{6}

% Current Problem number, should always start at 1
\newcounter{problemno}
\setcounter{problemno}{1}

%%
%% problem and bonus environment
%%
\newcounter{probcalc}
\newcommand{\problem}[2]{
	\pagebreak[2]
	\setcounter{probcalc}{#2}
	~\\
	{\large \textbf{Problem {\arabic{assignmentno}}.{\arabic{problemno}}} \hspace{0.2cm}\textit{#1}} \refstepcounter{problemno}\vspace{2pt}\\}

\newcommand{\bonus}[2]{
	\pagebreak[2]
	\setcounter{probcalc}{#2}
	~\\
	{\large \textbf{Bonus Problem \textcolor{blue}{\arabic{assignmentno}}.\textcolor{blue}{\arabic{problemno}}} \hspace{0.2cm}\textit{#1}} \refstepcounter{problemno}\vspace{2pt}\\}

%% some counters  
\newcommand{\assignment}{\arabic{assignmentno}}

%% solution  
\newcommand{\solution}{\pagebreak[2]{\bf Solution:}\\}

%% Hyperref Setup
\hypersetup{pdftitle={Homework \assignment},
	pdfsubject={\coursename},
	pdfauthor={},
	pdfcreator={},
	pdfkeywords={Computer Architecture and Programming Languages},
	%  pdfpagemode={FullScreen},
	%colorlinks=true,
	%bookmarks=true,
	%hyperindex=true,
	bookmarksopen=false,
	bookmarksnumbered=true,
	breaklinks=true,
	%urlcolor=darkblue
	urlbordercolor={0 0 0.7}
}

\begin{document}
	\coursename \hfill Course: \courseno\\
	Jacobs University Bremen \hfill \mytoday\\
	{Desar Mejdani}\hfill
	\vspace*{0.3cm}\\
	\begin{center}
		{\Large \sheettitle{} {\assignment}\\}
	\end{center}
	
\problem{}{0}
\solution
Only 5 bits are needed because there are in total 32 registers. Thus, the number of needed registers is 5. With 4 bits, it would not be possible to represent all of them. With 6, there would be a bit which would never be used. We would waste a bit.


\problem{}{0}
\solution
\begin{enumerate}[label=(\alph*)]
	\item 
	\begin{verbatim}
	sub $t2, $t0, $t1 # add t0 and t1 and store at t2
	\end{verbatim}
	
		\item 
	\begin{verbatim}
	lw $s2,4($s1) # load s1 + 4 into s2
	\end{verbatim}
	
\end{enumerate}

\problem{}{0}
\solution

\begin{enumerate}[label=(\alph*)]
	\item 
	\begin{verbatim}
	000000 01000 01001 01010 00000 100010
	\end{verbatim}

	\item 
	\begin{verbatim}
	100011 10001 10010 0000000000000100
	\end{verbatim}	
	
\end{enumerate}

	
\problem{}{0}
\solution
\begin{verbatim}slt $t2, $t0, $t1\end{verbatim} 
sets \$t2 to 1 if \$t0 is less than \$t1, which is true. \\
\begin{verbatim}beq $t2, $0, ELSE\end{verbatim} 
branches to ELSE if \$t2 and \$0 are equal, which is false. \\
\begin{verbatim}j DONE\end{verbatim} 
jumps to the end of the program.\\
Final value of \$t2 is 1 = 0000\,0000\,0000\,0000\,0000\,0000\,0000\,0001.

	
\problem{}{0}
\solution
\begin{verbatim}
lw $t0, 24($s0) 	# load A[6] into t0
add $t0, $t0, $s1	# add t0 and s1 and store at t0
sw $t0, 24($s0)		# save t0 at A[6]
\end{verbatim}

\problem{}{0}
\solution
\begin{verbatim}
lui $s4, 35 		# load the upper 16 bits
ori $s4, $s4, 35	# or immediate s4 with 35 and store at s4
\end{verbatim}

\newpage
\problem{}{0}
\solution
\begin{verbatim}
li $t0, 0 # load immediate 0 at t0
li $t1, 8 # load immediate 8 at t1

LOOP:
		beq $t1, $t0, EXIT # check if i==8, if true exit
		addi $s0, $s0, 4   # add 4 to s0 and store at s0
		addi $t0, $t0, 1   # i++
		j LOOP			   # repeat
EXIT: ...
\end{verbatim}
	
\end{document}

