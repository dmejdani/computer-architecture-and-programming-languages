\documentclass[a4paper]{article}
\usepackage[pdftex]{hyperref}
\usepackage[latin1]{inputenc}
\usepackage[english]{babel}
\usepackage{a4wide}
\usepackage{amsmath}
\usepackage{amssymb}
\usepackage{algorithmic}
\usepackage{algorithm}
\usepackage{ifthen}
\usepackage{listings}
% move the asterisk at the right position
\lstset{basicstyle=\ttfamily,tabsize=4,literate={*}{${}^*{}$}1}
%\lstset{language=C,basicstyle=\ttfamily}
\usepackage{moreverb}
\usepackage{palatino}
\usepackage{multicol}
\usepackage{tabularx}
\usepackage{comment}
\usepackage{verbatim}
\usepackage{color}

%% personal packages
\usepackage[super]{nth}
\usepackage{enumitem}
\usepackage[nodayofweek]{datetime}
\usepackage{circuitikz}
\usepackage{karnaugh-map}

%% new date format
\newdateformat{mydate}{\twodigit{\THEDAY}{ }\shortmonthname[\THEMONTH], \THEYEAR}

%% pdflatex?
\newif\ifpdf
\ifx\pdfoutput\undefined
\pdffalse % we are not running PDFLaTeX
\else
\pdfoutput=1 % we are running PDFLaTeX
\pdftrue
\fi
\ifpdf
\usepackage{graphicx}
\else
\usepackage{graphicx}
\fi
\ifpdf
\DeclareGraphicsExtensions{.pdf, .jpg}
\else
\DeclareGraphicsExtensions{.eps, .jpg}
\fi

\parindent=0cm
\parskip=0cm

\setlength{\columnseprule}{0.4pt}
\addtolength{\columnsep}{2pt}

\addtolength{\textheight}{5.5cm}
\addtolength{\topmargin}{-26mm}
\pagestyle{empty}

%%
%% Sheet setup
%% 
\newcommand{\coursename}{Computer Architecture and Programming Languages}
\newcommand{\courseno}{CO20-320241}

\newcommand{\sheettitle}{Homework}
\newcommand{\mytitle}{}
\newcommand{\mytoday}{\today}


% Current Assignment number
\newcounter{assignmentno}
\setcounter{assignmentno}{9}

% Current Problem number, should always start at 1
\newcounter{problemno}
\setcounter{problemno}{1}

%%
%% problem and bonus environment
%%
\newcounter{probcalc}
\newcommand{\problem}[2]{
	\pagebreak[2]
	\setcounter{probcalc}{#2}
	~\\
	{\large \textbf{Problem {\arabic{assignmentno}}.{\arabic{problemno}}} \hspace{0.2cm}\textit{#1}} \refstepcounter{problemno}\vspace{2pt}\\}

\newcommand{\bonus}[2]{
	\pagebreak[2]
	\setcounter{probcalc}{#2}
	~\\
	{\large \textbf{Bonus Problem \textcolor{blue}{\arabic{assignmentno}}.\textcolor{blue}{\arabic{problemno}}} \hspace{0.2cm}\textit{#1}} \refstepcounter{problemno}\vspace{2pt}\\}

%% some counters  
\newcommand{\assignment}{\arabic{assignmentno}}

%% solution  
\newcommand{\solution}{\pagebreak[2]{\bf Solution:}\\}

%% Hyperref Setup
\hypersetup{pdftitle={Homework \assignment},
	pdfsubject={\coursename},
	pdfauthor={},
	pdfcreator={},
	pdfkeywords={Computer Architecture and Programming Languages},
	%  pdfpagemode={FullScreen},
	%colorlinks=true,
	%bookmarks=true,
	%hyperindex=true,
	bookmarksopen=false,
	bookmarksnumbered=true,
	breaklinks=true,
	%urlcolor=darkblue
	urlbordercolor={0 0 0.7}
}

\begin{document}
\coursename \hfill Course: \courseno\\
Jacobs University Bremen \hfill \mytoday\\
{Desar Mejdani}\hfill
\vspace*{0.3cm}\\
\begin{center}
	{\Large \sheettitle{} {\assignment}\\}
\end{center}

\problem{}{0}
\solution
\begin{enumerate}[label=(\alph*)]
	\item Because in a single-cycle datapath the PC has to be incremented to the next address no matter the instruction, as each instruction takes one cycle.
	
	\item Because in a multi-cycle datapath one instruction can have several cycles, and the PC has to be incremented only once for each instruction.
\end{enumerate}

\problem{}{0}
\solution
\textbf{Note:} Active lines are represented by blue lines. Active selectors are represented by red circles.

\begin{enumerate}[label=(\alph*)]
\item 
	
\begin{figure}[!ht]
	\centering
	\includegraphics[scale=.5]{pic1.png}
	\caption{Datapath for part for the 'add' instruction}
\end{figure}

\begin{figure}[!ht]
	\centering
	\includegraphics[scale=.4]{pic2.png}
	\caption{Datapath for part the 'lw' instruction}
\end{figure}
\newpage

Values for the control lines:


\begin{table}[!ht]
	\centering
	\begin{tabular}{|c|c|c|c|c|c|c|c|c|}
		\hline
		Instruction & RegDst & ALUSrc & Memto Reg & Reg Write & Mem Read & Mem Write & Branch & ALUOp \\ \hline
		add         & 1      & 0      & 0         & 1         & 0        & 0         & 0      & 10    \\ \hline
		lw          & 0      & 1      & 1         & 1         & 1        & 0         & 0      & 00    \\ \hline
	\end{tabular}
\end{table}


\item 
\begin{enumerate}
	\item R-type instructions.\\
	 During the 'add' instruction the ALU will perform the sum of the values of the two registers. The control sequence sent to ALU is 0010 which will make the ALU perform the addition.
	
	\item I-type instruction.\\
	During the instruction such as 'lw' the ALU will calculate the correct address by adding the offset to the base address. 
\end{enumerate}
\end{enumerate}
\end{document}

