\documentclass[a4paper]{article}
\usepackage[pdftex]{hyperref}
\usepackage[latin1]{inputenc}
\usepackage[english]{babel}
\usepackage{a4wide}
\usepackage{amsmath}
\usepackage{amssymb}
\usepackage{algorithmic}
\usepackage{algorithm}
\usepackage{ifthen}
\usepackage{listings}
% move the asterisk at the right position
\lstset{basicstyle=\ttfamily,tabsize=4,literate={*}{${}^*{}$}1}
%\lstset{language=C,basicstyle=\ttfamily}
\usepackage{moreverb}
\usepackage{palatino}
\usepackage{multicol}
\usepackage{tabularx}
\usepackage{comment}
\usepackage{verbatim}
\usepackage{color}

%% personal packages
\usepackage[super]{nth}
\usepackage{enumitem}
\usepackage[nodayofweek]{datetime}
\usepackage{circuitikz}
\usepackage{karnaugh-map}

%% new date format
\newdateformat{mydate}{\twodigit{\THEDAY}{ }\shortmonthname[\THEMONTH], \THEYEAR}

%% pdflatex?
\newif\ifpdf
\ifx\pdfoutput\undefined
\pdffalse % we are not running PDFLaTeX
\else
\pdfoutput=1 % we are running PDFLaTeX
\pdftrue
\fi
\ifpdf
\usepackage{graphicx}
\else
\usepackage{graphicx}
\fi
\ifpdf
\DeclareGraphicsExtensions{.pdf, .jpg}
\else
\DeclareGraphicsExtensions{.eps, .jpg}
\fi

\parindent=0cm
\parskip=0cm

\setlength{\columnseprule}{0.4pt}
\addtolength{\columnsep}{2pt}

\addtolength{\textheight}{5.5cm}
\addtolength{\topmargin}{-26mm}
\pagestyle{empty}

%%
%% Sheet setup
%% 
\newcommand{\coursename}{Computer Architecture and Programming Languages}
\newcommand{\courseno}{CO20-320241}

\newcommand{\sheettitle}{Homework}
\newcommand{\mytitle}{}
\newcommand{\mytoday}{\today}


% Current Assignment number
\newcounter{assignmentno}
\setcounter{assignmentno}{5}

% Current Problem number, should always start at 1
\newcounter{problemno}
\setcounter{problemno}{1}

%%
%% problem and bonus environment
%%
\newcounter{probcalc}
\newcommand{\problem}[2]{
	\pagebreak[2]
	\setcounter{probcalc}{#2}
	~\\
	{\large \textbf{Problem {\arabic{assignmentno}}.{\arabic{problemno}}} \hspace{0.2cm}\textit{#1}} \refstepcounter{problemno}\vspace{2pt}\\}

\newcommand{\bonus}[2]{
	\pagebreak[2]
	\setcounter{probcalc}{#2}
	~\\
	{\large \textbf{Bonus Problem \textcolor{blue}{\arabic{assignmentno}}.\textcolor{blue}{\arabic{problemno}}} \hspace{0.2cm}\textit{#1}} \refstepcounter{problemno}\vspace{2pt}\\}

%% some counters  
\newcommand{\assignment}{\arabic{assignmentno}}

%% solution  
\newcommand{\solution}{\pagebreak[2]{\bf Solution:}\\}

%% Hyperref Setup
\hypersetup{pdftitle={Homework \assignment},
	pdfsubject={\coursename},
	pdfauthor={},
	pdfcreator={},
	pdfkeywords={Computer Architecture and Programming Languages},
	%  pdfpagemode={FullScreen},
	%colorlinks=true,
	%bookmarks=true,
	%hyperindex=true,
	bookmarksopen=false,
	bookmarksnumbered=true,
	breaklinks=true,
	%urlcolor=darkblue
	urlbordercolor={0 0 0.7}
}

\begin{document}
	\coursename \hfill Course: \courseno\\
	Jacobs University Bremen \hfill \mytoday\\
	{Desar Mejdani}\hfill
	\vspace*{0.3cm}\\
	\begin{center}
		{\Large \sheettitle{} {\assignment}\\}
	\end{center}
	
	\problem{}{0}
	\solution
	\textbf{Note:} The down-arrow on top of the digits indicates the place where the carry is being applied.\\
	\textbf{Note:} Every number which does not have its base written next to it, it is meant as a number in base 10.
	\begin{enumerate}[label=(\alph*)]
		\item $14 + 37$\\
		$14 = 2^3 + 2^2 + 2^1 = 1110_2$\\
		$37 = 2^5 + 2^2 + 2^0 = 100101_2$\\
		
		\begin{tabular}{r}
			$1\overset{\downarrow}{0}\overset{\downarrow}{0}101$\\
			$+001110_2$\\
			\hline
			$110011_2$\\		
		\end{tabular}
		
		\item $12 - 27$ - Using 8-bit two's complement.\\
		
		$12 = 2^3 + 2^2 = 00001100_2$\\
		$-27 = - (2^4 + 2^3 + 2^1 + 2^0) = - (00011011_2) \implies \text{Flipping bits} 11100100 \implies \text{Adding one} = 11100101$\\
		Now we add them bit by bit:
		
		\begin{tabular}{r}
			$000\overset{\downarrow}{0}\overset{\downarrow}{1}100_2$\\
			$+11100101_2$\\
			\hline
			$11110001_2$\\		
		\end{tabular}
		
		$11110001_2 = - (00001110_2 + 1) = -15$
		
		\item $69 + 58$ - Using BCD code.\\
		
		$69 = 0110\,1001_{BCD}$\\
		$58 = 0101\,1000_{BCD}$\\
		
		\begin{tabular}{r}
			$\overset{\downarrow}{0}110\,1001_2$\\
			$+0101\,1000_2$\\
			\hline
			$1011\,10001_2$
		\end{tabular}
		\quad Both parts of the number are not valid BCD code (higher than 9). 
		
		$10001_{BCD} = 17_{10} \implies \text{Adding 6} = 23_{10} \implies \text{Subtracting 16} + \text{Carry of 1 to the next number} = 7 + \text{Carry of 1 to the next number} = 0111_{BCD} + \text{Carry of 1 to the next number}$\\
		$101\overset{\downarrow}{1}_{BCD} = 1100_{BCD} = 12 \implies \text{Adding 6} = 18 \implies \text{Subtracting 16 + carry of 1 to the next number} = 2 \text{carry of 1} = 0001\,0010_{BCD}$\\
		Result: $0001\,0010\,0111_{BCD} = 127_{10}$
		
		\item $275 + 642$ - Using BCD code\\
		$275 = 0010\,0111\,0101_{BCD}$\\
		$642 = 0110\,0100\,0010_{BCD}$\\
		
		\begin{tabular}{r}
			$\overset{\downarrow}{0}\overset{\downarrow}{0}10\,\overset{\downarrow}{0}111\,0101$\\
			$0110\,0100\,0010$\\
			\hline
			$1000\,1011\,0111$
		\end{tabular}
		\quad The second BCD number is not a valid one\\
		$1011_2 = 11_{10} \implies \text{add 6} = 17 \implies \text{subtract 16, and carry 1} = 1 \text{+ carry of 1} = 0001_2 \text{+ carry of 1}$\\
		Result: $100\overset{\downarrow}{0}\,0001\,0111_{BCD} = 1001\,0001\,0111_{BCD} = 917_{10}$
		
		\item $6AF + 23C$ - Using hexadecimal representation\\
		
		\begin{tabular}{r}
			$6AF$\\
			$+23C$\\
			\hline
			...\\
		\end{tabular}
		
		$F+C = 15 + 12 = 27 \implies 27 - 16 + \text{carry 1} = B + \text{carry 1}$\\
		$A + 3 = 10 + 3 = 13 = D \implies \text{+ 1 carry from previous digit} = E$\\
		$6 + 2 = 8$\\
		\begin{tabular}{r}
			$6AF$\\
			$+23C$\\
			\hline
			$8EB_{16}$\\
		\end{tabular}
		
		\item $594 - 3A8$ - Using hexadecimal representation\\
		First, we need to find the 2's complement of 3A8.\\
		
		\begin{tabular}{r}
			$FFF$\\
			$-3A8$\\
			\hline
			$C57$\\
		\end{tabular}
		
		Now we can add the numbers:
		
		\begin{tabular}{r}
			$594$\\
			$+C57$\\
			\hline
			$11EB$\\
		\end{tabular}
		To get the correct result we add 1 and disregard the MSD.
		
		Result: $1EC_{16}$
	\end{enumerate}
	
	\problem{}{0}
	\solution
	\begin{enumerate}[label=(\alph*)]
		\item $\mathrm{a = b + c}$
		\begin{verbatim}
		add $t0, $s0, $s1 #add s0 with s1 and store pass it to t0
		\end{verbatim}
		
		\item $\mathrm{a = b - d + c}$
		\begin{verbatim}
		sub $t0, $s0, $s2 #subtract s2 from d0 and pass it to t0
		add $t0, $t0, $s1 #add t0 to s1 and pass it to t0
		\end{verbatim}
		
		\item $a = 3 * b$
		\begin{verbatim}
		add $t0, $s0, $s0 #add s0 to itself and pass to t0
		add $t0, $to, $s0 #add s0 to t0 and pass it to t0
		\end{verbatim}
		
		\item $a = (1 + b) * 2$
		\begin{verbatim}
		add $t0, 1, $s0 #add 1 to s0 and pass it to t0
		add $t0, $t0, $t0 #add t0 to t0 and pass it to t0
		\end{verbatim}
		
	\end{enumerate}
	
	\problem{}{0}
	\solution
	\begin{enumerate}[label=(\alph*)]
		\item 	
		\begin{verbatim}
		add $t0, $s0, $s1 #add s0 with s1 and store pass it to t0
		\end{verbatim}
		
		\begin{table}[!ht]
			\centering
			\begin{tabular}{|l|l|l|l|l|l|}
				\hline
				\textbf{op} & \textbf{rs}/s0 & \textbf{rt}/s1 & \textbf{rd}/t0 & \textbf{shamt} & \textbf{funct} \\ \hline
				000000      & 10000       & 10001       & 01000       & 00000           & 100000         \\ \hline
			\end{tabular}
		\end{table}
		
		Binary code: 000000 10000 10001 01000 00000 100000
		
		\item
		\begin{verbatim}
		sub $t0, $s0, $s2 #subtract s2 from d0 and pass it to t0
		add $t0, $t0, $s1 #add t0 to s1 and pass it to t0
		\end{verbatim}
		
		\begin{table}[!ht]
			\centering
			\begin{tabular}{|l|l|l|l|l|l|}
				\hline
				\textbf{op} & \textbf{rs} & \textbf{rt} & \textbf{rd} & \textbf{shamt} & \textbf{funct} \\ \hline
				000000      & 10000/s0       & 10010/s2       & 01000/t0      & 00000          & 100010         \\ \hline
				000000      & 01000/t0       &10001/s1        & 01000/t0       & 00000          & 100000         \\ \hline
			\end{tabular}
		\end{table}
		
		Binary code: 000000 10000 10010 01000 00000 100010 000000 01000 10001 01000 00000 100000
		
	\end{enumerate}
	\newpage
	
	\problem{}{0}
	\solution
	$\mathrm{B[5] = A[4] + A[2]}$
	
	\begin{verbatim}
	lw $t0, 16($s0) 	#loading A[4] to $t0
	lw $t1, 8($s0) 		#loading A[2] to $t1
	add $t0, $t0, $t1	#adding the values and storing to $t0
	sw $t0, 20($s1) 	#saving the result into B[5]
	\end{verbatim}
	
	\problem{}{0}
	\solution
	$\mathrm{B[x] = A[x+7] + A[x+2]}$
	
	\begin{verbatim}
	# comments are describing in byte addressing
	add $t1, $t0, 7   # Adding 7 to x
	add $t1, $t1, $t1 # 2*(x+7)
	add $t1, $t1, $t1 # 4*(x+7)
	add $t1, $t1, $s0 # 4*(x+7) + base of A
	lw $t1, 0($t1)    # t1 = A[4*(x+7)]
	
	add $t2, $t0, 2   # Adding 2 to x
	add $t2, $t2, $t2 # 2*(x+2)
	add $t2, $t2, $t2 # 4*(x+2)
	add $t2, $t2, $s0 # 4*(x+2) + base of A
	lw $t2, 0($t2)    # t2 = A[4*(x+2)]
	
	add $t1, $t1, $t2 # t1 = t1 + t2
	
	add $t0, $t0, $t0 # t0 = 2*x
	add $t0, $t0, $t0 # t0 = 4*x
	add $t0, $t0, $s1 # t0 = 4*x + base of B
	sw $t1, 0($t0)	   # B[4*x] = t1	
	\end{verbatim}
	
	\problem{}{0}
	\solution
	The change on the number of general purpose registers can affect the total number of bits for on the addi instruction.\\
	The opcode length does not get affected from the change, still 6 bits.
	First register only needs 4 bits to represent the total of 12 registers.
	Second register also only needs 4 bits for the same reason.\\
	The constant value part does not get affected from the change, still 16 bits.
	
\end{document}

